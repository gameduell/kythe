\documentclass[10pt,letterpaper]{article}
\usepackage[margin=1in]{geometry}
\usepackage{hyperref}
\usepackage{relsize}
\usepackage{xspace}
\newcommand{\mailto}[1]{\href{mailto:#1}{\nolinkurl{#1}}}
% The International Standard Way to Typeset "C++"
% (https://github.com/cplusplus/draft/blob/master/source/macros.tex)
\newcommand{\Rplus}{\protect\hspace{-.1em}\protect\raisebox{.35ex}{\smaller{\smaller\textbf{+}}}}
\newcommand{\Cpp}{\mbox{C\Rplus\Rplus}\xspace}
\newcommand{\Cyy}{\mbox{C\Rplus\Rplus}14\xspace}
\begin{document}
\title{Indexing Large, Mixed-Language Codebases}
\author{James Dennett, Luke Zarko\\
{\small\mailto{jdennett@google.com}, \mailto{zarko@google.com}}\\
\\
Google, Inc.}
\date{}
\maketitle
\thispagestyle{empty}

The Kythe project aims to establish open data formats and protocols for
interoperable developer tools. In this talk, we will introduce the Kythe model
as it applies to \Cyy, concentrating on features required for generating
cross-references. We will discuss how the Clang front-end was instrumental in
developing an indexing tool that produces Kythe data describing \Cpp source
code. We will also give an overview of a reference pipeline that consumes code
and build information to produce a navigable index. This index, when combined
with Kythe data from other indexers, allows developers to explore code from
projects composed of many different languages.

Our overall goal is to loosen the coupling between programming languages and
the software that supports them. With the Kythe abstractions, a single tool
may handle a corpus containing code written in any number of languages, some
of which the tool's author may never have encountered. Language and compiler
authors who produce Kythe data immediately benefit from global support rather
than having to implement a custom version of each tool in turn. In this way
Kythe reduces the product of M languages by N tools to a sum.

We are interested in collaborating with others who are also indexing code with
Clang. Of course, we would also like to work with developers who can build
indexers for other programming languages as well. The Kythe model is derived
from the format used internally at Google to cross-reference a very large
mixed-language source repository. It is designed to scale in corpus size and
diversity, in language count, and in variety of tooling. Development along any
of these axes benefits all participants in the community.

\end{document}
